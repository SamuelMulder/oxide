\documentclass[]{article}   

\usepackage[margin=2.5cm]{geometry}
\begin{document}

\begin{center}
\underline{\LARGE Equivalence Database Formatting Guide}\\
\end{center}
\vspace{5 mm}
The equivalence database is located in \emph{core\/libraries\/equivalence\_database}.  It is used in conjunction with the alg\_ident plugin currently located in \emph{modules\/extractors\_dev\/alg\_ident}.
Its purpose is to automatically offer alternatives to common assembly instructions written in the templates used by alg\_ident.  For instance, a rotate-right (ror) can be rewritten as a series of shifts
followed by an or operation.  Rather than enumerating all the possibilities within each template, the equivalence database offers a fast alternative. \\

\noindent When adding new entries into the database, the following formatting standards should be followed:\\ \\
1) Each new instruction entry should be on its own line.  The line should contain the instruction mnemonic, a colon, and a comma-delimited list of instructions needed to comprise the equivalence\\
\emph{EX: ror: shr, shl, or}\\
2) Each line between that instruction entry and the next should be another representation of that instruction\\
3) Multiple instruction sequences should be separated by semicolons\\
4) Terminal variables can be used to represent unknown values that need to be the same in certain positions:\\
\emph{EX: shl C 32-B; shr A B; or C A}\\
5) Mathematical expressions such as the one in the previous example will be automatically evaluated\\
6) All terminal variables should be single capital letters (A, B, C, etc.), and everything else should be lowercase\\
7) Blank lines are ignored\\
8) White space is ignored as long as it separates variables and instructions\\



  

\end{document}
